\documentclass[amsmath,amssymb]{revtex4}
\usepackage{multirow}
\renewcommand{\multirowsetup}{\centering}

\usepackage{fancyheadings}
\pagestyle{fancy}
\chead{Satish Desai - Curriculum Vitae}
%\topmargin 0.0in
\def\dzero{D\O}
\def\cluedo{CLuED\O}

\begin{document}
\thispagestyle{empty}
\begin{center}
%\rule{7.0in}{0.02cm}
\begin{table}[h]
\begin{tabular}{llcl}
\hline
\large E-mail: & \large desai@physics.umn.edu & & \large Tate Lab Room 116\\ 
\large Office: & \large +1-612- 624-1020        & \multirow{3}{3.0in}{{\huge{\bf Satish Desai}}\\} & \large 116 Church Street S.E  \\
\large Mobile: & \large +1-630-712-6133         & & \large Minneapolis, MN 55455 \\
\hline
\end{tabular}
\end{table}
\end{center}

%%\noindent{\Large \underline{Date of Birth:}}\large~July 19, 1975\\

\noindent{\Large \underline{Education:}}\\
\begin{list}{\labelitemi}
            {\setlength{\itemsep}{0.0in}\setlength{\parsep}{0.0in}
             \addtolength{\parskip}{-0.5in}}
\item Ph.D. in Elementary Particle Physics, Stony Brook University, Stony Brook, NY (2006)\\
   Dissertation Supervisor: John Hobbs\\
   Dissertation: {\it A Search for the Production of Technicolor Particles at the \dzero\ Detector}
\item Master of Arts, Physics, Stony Brook University, Stony Brook, NY (1998)
\item Bachelor of Arts, Physics, Cum Laude,  Franklin and Marshall College , Lancaster, PA (1997)\\
  Honors Thesis Adviser: Russel Kauffman \\
  Honors Thesis: {\it Production of a Higgs Pseudoscalar Plus Two Jets in Hadronic Collisions}
\end{list}
\vspace{\baselineskip}

\noindent{\Large \underline{Honors and Awards:}}\\
\begin{list}{\labelitemi}
            {\setlength{\itemsep}{0.0in}\setlength{\parsep}{0.0in}
             \addtolength{\parskip}{-0.5in}}
\item Kershner Scholar, 1996, 1997, awarded for academic achievement in physics.
\item Member Sigma Pi Sigma physics honors society since 1996.
\item Presidential Scholar, 1993 - 1997, awarded for general academic achievement.
\end{list}
\vspace{\baselineskip}

\noindent{\Large \underline{Positions:}} \\
\begin{list}{\labelitemi}
            {\setlength{\itemsep}{0.0in}\setlength{\parsep}{0.0in}
             \addtolength{\parskip}{-0.5in}}
\item Research Associate, University of Minnesota, Minneapolis, MN (2012 - present)
\item Research Associate, Fermi National Accelerator Laboratory, Batavia, IL (2006 - 2012)
\item Graduate Research Assistant, Stony Brook University, Stony Brook, NY %\\
\item Teaching Assistant, Stony Brook University, Stony Brook, NY (1997-1998) %\\
\item Undergraduate Research Assistant, Franklin and Marshall College, Lancaster, PA (1996 - 1997)
\item Undergraduate Teaching Assistant, Franklin and Marshall College, Lancaster, PA (1995 - 1997)
\item Undergraduate Physics Tutor, Franklin and Marshall College, Lancaster, PA (1995 - 1997)
\end{list}
\vspace{\baselineskip}

%\noindent{\Large \underline{Research Interests:}} Electroweak symmetry breaking,
%Precision electroweak measurements, QCD, Grid computing, Detector development,
%construction and operations. \\

%%\noindent{\Large \underline{Professional Affiliations:}} American Physical Society \\

%%\pagebreak
\noindent{\Large \underline{Experience:}} \\

\noindent{\bf NO$\nu$A Data Processing} \\
PCHits
MC Reco
General Submitter

\noindent{\bf NO$\nu$A Detector Quality Assurance} \\
Factory QA
Fardet QA

\noindent{\bf Searches for the Source of Electroweak Symmetry Breaking} \\
\indent From 2010 to 2012, I was a co-convener of the
\dzero~low mass Higgs group.  In this capacity, I directed the search
for a standard model Higgs boson decaying to $b\bar{b}$ pairs.  This
final state accounts for $\sim$60\% of the decays for a standard model Higgs boson
with a mass of $125$ GeV.  During my term, we released several preliminary
updates of all major search channels.  To validate our analysis techniques, we combined these analyses
to obtain 3.3$\sigma$ evidence for $VZ$ production in the $\ell\nu b{\bar b}$, 
$\ell\ell b{\bar b}$ and $\nu\nu b{\bar b}$ final states, and subsequently
4.6$\sigma$ evidence in conjunction with CDF.    I was the primary analyzer for
the $\ell\ell b{\bar b}$ contribution to this result, where I trained the final random
forest discriminant (RF) and developed selection criteria to reject a
poorly modeled kinematic region.  I guided the combined result through collaboration review to be shown at the
HCP 2011 symposium.

I led the effort to publish updated results using the full Run II data
set of 9.7 fb$^{-1}$ in the summer of 2012.  In combination with CDF
we obtained evidence, with a significance of 3.1$\sigma$ (after correcting for
the look-elsewhere-effect), for a new particle decaying to $b{\bar b}$,
and produced in association with a $W$ or $Z$ boson with a mass in
the range between 120 and 135 GeV.  This excess is consistent with the new boson
observed in Higgs searches by the CMS and ATLAS collaborations, and provided
the first evidence that this new particle couples to $b$-quarks, as expected for a SM
Higgs boson.

I also led the $ZH\to\ell^+\ell^-b{\bar b}$ analysis team, which
includes four postdocs and four students.  I coordinated the analysis
of four separate channels, including two specialized channels that
extend the acceptance for $Z\to\ell\ell$ decays.  I was one of the
original developers of the analysis software, and the primary
maintainer of that package.  In that role, I streamlined the
analysis flow; wrote three batch-submission frameworks for the more
computing intensive portions of the analysis; and improved the
usability and documentation so that it could be easily adopted by new
members of the analysis team.

I participated in the design of a normalization scheme based on
simultaneous fit to the dilepton mass spectra from all channels that
suppresses the impact of systematic uncertainties and serves as a
powerful check on the internal consistency of the analysis.  I trained
the final RF used to separate the signal from background and
supervised a student in the development of a dedicated RF to reject the
background from $t{\bar t}$.  I was also responsible for executing the
statistical analysis to obtain the final results.  These efforts have
resulted in two publications in Physical Review Letters using a data sets
of 4.2 fb$^{-1}$ and a 9.7 fb$^{-1}$, as well as several preliminary updates of
the analysis in the intervening time.

My dissertation was based on a search for technicolor in the
$\rho_{TC} \to W \pi_{TC} \to \mu\nu b{\overline b}/ b{\overline c}$
channel.  Technicolor is an alternative to the standard model Higgs
mechanism, in which a new strong interaction dynamically breaks
electroweak symmetry.  Because this model contains no fundamental
scalars, it is not susceptible to the large radiative corrections that
trouble the standard theory.  In addition to determining the initial
selection requirements and corrections to the background model, I
designed a random grid search
to obtain separately optimized analyses for 20 mass hypotheses.\\

\noindent{\bf Internal Review of $t{\bar t}$ Cross Sections}\\
\indent Since 2008, I have been an active participant in the internal
review boards for the \dzero~ $t{\bar t}$ cross section measurements.
In this capacity, I have also reviewed several results in similar
final states, including searches for scalar top quarks,
charged Higgs bosons from $t{\bar t}$ decays, and $t{\bar t}H$ production.\\

\noindent{\bf Higgs Cross Section Calculations}\\
\indent I calculated the production cross sections for a Higgs boson
plus additional jets in gluon fusion processes.  This work resulted in
two papers, one for a standard model Higgs and one for a pseudoscalar
Higgs boson, as predicted by multi-Higgs doublet models such as
supersymmetry.  The latter formed the basis for my undergraduate honors
thesis. \\

\noindent{\bf Tracking System Operations at \dzero} \\
\indent I was co-leader of the operations group for the \dzero~silicon
microstrip tracker (SMT) from December 2006 through July 2008.  I streamlined
operations and eliminated the leading causes of deadtime from the SMT.
These efforts allowed the collaboration to reduce the number of shifters
needed to operate \dzero~from five to four while achieving a data taking
efficiency of 90\%.  I developed a tool to monitor the hit efficiency
for individual SMT sensors in real time, allowing the rapid diagnosis and
resolution of readout problems on a granular level.  

Early in my term, we identified one of the primary failure modes for
the readout chips and developed a novel workaround that enabled us to
recover failed devices using an alternative power supply and
distribution system.  In the summer of 2007 I planned and managed the
implementation of this workaround by a team of physicists, engineers
and technicians on a tight schedule.  I developed a prioritized list
of devices that could potentially benefit from this workaround and
devised a testing procedure to confirm this hypothesis. I further
identified an unexpected vulnerability that the workaround induced,
and supervised the development of both hardware and software based
protection systems to guard against it.  I was deeply
involved in subsequent shutdowns in 2008 and 2009 where we expanded
the usage of the workaround.  During these shutdowns, I was
instrumental in the recovery of even more devices by remapping the
readout and high voltage distribution for sensors with failed readout
and control electronics.  These efforts increased the fraction of
active SMT readout channels to its highest level in \dzero~history,
and established a new trend of stability in SMT operations that have
benefited all analyses at \dzero.  I served as an on-call expert for
the SMT from 2006 through the end of \dzero~operations in 2011.  I
have also served as an on-call expert for the \dzero~central fiber
tracker.  In both the SMT and the CFT groups, I was responsible for
training shifters to monitor
routine data-taking and spot problems for these detectors.\\

%The \dzero\ tracking system consists of a silicon microstrip tracker
%(SMT), a central fiber tracker (CFT) as well as central and forward preshower detectors.
%Both are essential to the physics program at \dzero.  I was co-leader of the SMT operations
%group from December 2006 to July 2008, and remain an active member of that group.  I managed
%the silicon tracker repair activities during a long shutdown in the summer of 2007.  This 
%resulted in the recovery of a subtantial number of disabled channels.  I was deeply involved
%in subsequent shutdowns in October 2008 and the summer of 2009.  As a consequence of these
%efforts, the SMT has more active channels than at any previous time.  I currently serve as an
%on-call expert for the SMT, and have previously served as an on-call expert for the CFT.  I am
%also responsible for training shifters to monitor routine data-taking and spot problems for these
%detectors.
%
%Additionally, as a graduate student, I served as an on-call expert for both the CFT and the
%preshowers detectors. As such, I was responsible for the rapid diagnosis and resolution of
%problems with these detectors and their associated readout electronics.  I also
%trained shifters to monitor routine data-taking and spot problems for the same.\\

\noindent{\bf \dzero\ Central Track Trigger Commissioning}\\
%\indent The high luminosities at the Tevatron preclude the possibility of recording anything but
%a small fraction of the collisions at the \dzero\ detector.  A
%three-tiered trigger system is used to intelligently select which
%events to save for later analysis.  The Level 1 central track trigger
%(CTT) uses hits from the CFT to provide tracking information to the
%first tier of the trigger system.  Because of the small amount of time
%(4.2 $\mu$s) available for the initial trigger decision, the CTT is
%implemented in firmware running on custom made digital front-end
%boards.
%
%My initial responsibilities included development of trigger algorithms
%for the forward preshower.  However, the various components of the CTT
%faced the common task of synchronizing data from multiple sources and
%storing it for later transmission to the data-acquisition system.
%Developing algorithms that solved these problems and functioned
%reliably at the 53 MHz bus-speed proved to be an extremely difficult
%problem to solve.  I co-wrote and tested the prototype firmware used
%throughout the CTT for input synchronization and re-transmission.
%
%During the commissioning of the CTT, I identified and analyzed trigger
%timing issues.  Resolving these problems required the rapid production
%and testing of new revisions of the communication firmware in
%simulation, on a test bench and in the production system.  The lessons
%learned during this process provided valuable inputs to the design of
%the final synchronization algorithms.\\
\indent I played a key role in the commissioning of the \dzero~central 
track trigger.  I developed the firmware responsible for
receiving and synchronizing data from multiple sources, and storing it
for later transmission to the data acquisition system.  In this
capacity, I identified and analyzed trigger timing issues.  Resolving
these problems required the rapid production and testing of new
revisions of the firmware in simulation, on a test bench, and in the
production system.\\

\noindent{\bf \dzero\ Forward Preshower Construction}\\
\indent During test beam studies of the forward preshower (FPS) and 
silicon microstrip tracker, I ran tests of the flex circuits used in
the prototype readout electronics for both subsystems and oversaw the
construction of these electronics for the FPS.  Later, I participated
in the design, construction and calibration of the LED based pulser
system for the forward preshower.  To this end, I developed data
acquisition and analysis software for the calibration test stand.  I
took part in the routing of the optical fibers connecting the FPS to
the readout electronics.  I also initiated and led the effort to
verify the fiber connectivity and mapping through use of the pulser system.\\

\noindent{\bf Linux System Administration}\\
\indent From 2001 - 2004 I was a member of a small team of core
administrators for the \cluedo\ Linux cluster.  During this period,
the \cluedo~cluster was adopted by the \dzero~collaboration as its
primary desktop computing solution, and grew to encompass more than
five hundred PCs and servers of widely varying configurations.  In
addition to standard desktop services, \cluedo\ supports the \dzero\
Run II software infrastructure and provides an interface to the
\dzero\ central batch servers; furthermore it supplies a batch system
of its own and access to the \dzero\ data storage tapes from selected
nodes.  As part of the administration team, I installed new systems,
promptly addressed the variety of issues that arise in such a large
cluster, and participated in the planning and execution of two
cluster-wide operating system upgrades.  I have also maintained a
number of other Linux systems on a smaller scale, including a number
linux desktops at Stony Brook, several laptops, as well as a desktop
with a software level RAID array that I built for my personal use.\\

%
%I was a member of a small team of core administrators for the cluster, which is based on the
%Scientific Linux 4 distribution.  In addition to setting up individual machines, administrators are
%required to respond promptly to the variety of problems that arise in such a large system;  I
%also participated in the planning and execution of two cluster-wide operating system upgrades. \\

\noindent{\bf \dzero~Advisory Council}\\
\indent In October 2010, I was elected by the collaboration to the
\dzero\ Advisory Council.  This is a seven member board that
communicate the concerns of the collaboration to the spokespersons,
and provides advice on important decisions. Issues that the committee
has considered include scientific goals, resource allocation and
prioritization (especially of computing resources), funding for
continued work on the Tevatron, the long term preservation of
\dzero~data and internal notes, and support for
the career advancement of younger physicists.\\

\noindent{\bf Teaching Experience}\\
\indent In the fall of 2014, I taught the ''Energy and the
Evironment'' introductory physics course for non-science majors at the
University of Minnesota.  I have several years of experience as a
teaching assistant in introductory physics laboratory courses, at both
Franklin and Marshall College and at Stony Brook. I have also served
as a tutor for both the first year physics courses and for the
intermediate Electricity and Magnetism course.

At the University of Minnesota, I organized the 2014 Quarknet
Masterclass for high school students.  In previous years, I served as
a scientist moderator for this program while at Fermilab.  At
Fermilab, I participated in the Ask-a-Scientist program and attend
question-and-answer sessions for tour groups.  I also serve as a tour
guide for the Fermilab's Saturday Morning Physics program.  I gave
tours of \dzero\ to groups of diverse backgrounds, ranging from high
school students to visiting scientists.  I served as a judge at the
Department of Energy sponsored 2009 SERCh
competition for undergraduate research. \\

%\newpage
\noindent{\Large \bf{Publications:}} \\

\noindent{\large \bf{Publications with the \dzero\ Collaboration}}: \\

I am an author on all Run II publications from the \dzero\ collaboration. A
complete list may be found at:
http://www-d0.fnal.gov/d0\_publications/d0\_pubs\_list\_bydate.html\\

\input svdpub.tex

%\input svdproc.tex
%
\noindent{\Large \bf{Selected Presentations:}}

\begin{enumerate}
\item University of Illinois at Chicago, HEP Seminar, September 2012:
``Evidence for a New Particle Produced in Association with Weak Bosons 
  and Decaying to $b\bar{b}$ Pairs in Higgs Searches at the Tevatron''
\item 36th International Conference on High Energy Physics, July 2012:
``Searches for $H\to b\bar{b}$ at \dzero''
\item Fermilab Joint Experimental-Theoretical Physics Seminar, March 2012:
``Latest Higgs Results from \dzero''
\item Les Rencontres de Physique de la Vallée d'Aoste, March 2012:
``Low Mass Higgs Searches at the Tevatron''
\item University of California at Davis, Colloquium, October 2011:
``Hunting the Higgs at the Tevatron''
\item 19th International Conference on Supersymmetry and Unification of Fundamental Interactions,
Fermilab, September 2011:
``Tevatron Combination of SM Higgs Searches and 4th Generation Limits''.
\item 23rd Rencontres de Blois, Blois France, June 2011:
``Search for Low Mass Higgs at the Tevatron.''
\item Indiana University HEP Seminar, November 2010:
``Hunting for the Higgs at \dzero''
\item 30th Physics in Collision Symposium, Karlsruhe, Germany.  September 2010:
``Higgs Searches at the Tevatron.''
\item The University of \dzero~Lecture Series, May 2010: ``Making Tracks at \dzero''
\item 20th Hadron Collider Physics Symposium, Evian, France. November 2009:
``Low Mass Higgs at the Tevatron.''
\item Pheno 2009 Symposium, Madison, Wisconsin, May 2009: 
``Search for $ZH\to e^+e^-b{\bar b}$ in $p\bar{p}$ Collisions at \dzero''
\item 17th International Workshop on Vertex Detectors, Uto Island, Sweden.  July 2008:
``Radiation Damage Study of the \dzero\ Silicon Microstrip Tracker''
\item European Physical Society Conference on High Energy Physics, Manchester, England.  July 2007:
``Searches for Resonant Higgs Production at \dzero.''
%\item Meeting of the American Physical Society, Dallas, TX.  April 2006:
%``A Search for Technicolor Production at \dzero.''
%\item Meeting of the American Physical Society, Dallas, TX.  April 2006:
%``A Search for WH Associated Production at \dzero.''
%\item Meeting of the American Physical Society, Denver, CO.  May 2004:
% ``$W(\to \mu\nu)b{\overline b}$ Production at \dzero.''
%\item Meeting of the American Physical Society, Denver, CO.  May 2004:
% ``$W(\to e\nu)b{\overline b}$ Production at \dzero.''
%\item Meeting of the American Physical Society, Philadelphia, PA.  April 2003:
% ``The Level 1 Central Track Trigger at \dzero.''
\end{enumerate}

\end{document}
