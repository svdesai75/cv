\documentclass[amsmath,amssymb]{revtex4}
\usepackage{multirow}
\usepackage[none]{hyphenat}
\renewcommand{\multirowsetup}{\centering}

\usepackage{fancyheadings}
\pagestyle{fancy}
\chead{Satish Desai - Curriculum Vitae}
%\topmargin 0.0in
\def\dzero{D\O}
\def\cluedo{CLuED\O}

\begin{document}
\thispagestyle{empty}
\begin{center}
%\rule{7.0in}{0.02cm}
\begin{table}[h]
\begin{tabular}{llcl}
\hline
\large E-mail: & \large desai@physics.umn.edu & & \large Tate Lab Room 148\\ 
\large Office: & \large +1-612-624-1020        & \multirow{3}{3.0in}{{\huge{\bf Satish Desai}}\\} & \large 116 Church Street S.E  \\
\large Mobile: & \large +1-630-712-6133         & & \large Minneapolis, MN 55455 \\
\hline
\end{tabular}
\end{table}
\end{center}

%%\noindent{\Large \underline{Date of Birth:}}\large~July 19, 1975\\

\noindent{\Large \underline{Education:}}\\
\begin{list}{\labelitemi}
            {\setlength{\itemsep}{0.0in}\setlength{\parsep}{0.0in}
             \addtolength{\parskip}{-0.5in}}
\item Ph.D. in Elementary Particle Physics, Stony Brook University, Stony Brook, NY (2006)\\
   Dissertation Supervisor: John Hobbs\\
   Dissertation: {\it A Search for the Production of Technicolor Particles at the \dzero\ Detector}
\item Master of Arts, Physics, Stony Brook University, Stony Brook, NY (1998)
\item Bachelor of Arts, Physics, Cum Laude,  Franklin and Marshall College , Lancaster, PA (1997)\\
  Honors Thesis Adviser: Russel Kauffman \\
  Honors Thesis: {\it Production of a Higgs Pseudoscalar Plus Two Jets in Hadronic Collisions}
\end{list}
\vspace{\baselineskip}

\noindent{\Large \underline{Evidence of Esteem:}}\\
\begin{list}{\labelitemi}
            {\setlength{\itemsep}{0.0in}\setlength{\parsep}{0.0in}
             \addtolength{\parskip}{-0.5in}}
\item ``First measurement of electron neutrino appearance in NOvA,'' will be featured as a Phys. Rev.
Lett editors suggestion (2016).
\item ``Tevatron Constraints  on Models of the Higgs Boson with Exotic Spin and Parity Using Decays to
        Bottom-Antibottom Quark Pairs,'' was featured as a  Phys. Rev. Lett editors suggestion (2015).
\item Selected to give special Fermilab Joint Experimental-Theoretical Seminar (Wine and Cheese)
      on new \dzero~Higgs results (March 2012)
\item  ``Evidence for a particle produced in association with weak bosons and decaying to a
        bottom-antibottom quark pair in Higgs boson searches at the Tevatron,'' (2012) is a TOPCITE 100+ paper, and
        was highlighted by Phys. Rev. Lett. as an editor's suggestion and a ``Physics Viewpoint Article''.
\item Elected by \dzero~Collaboration to Advisory Council in 2010
%\item Kershner Scholar, 1996, 1997, awarded for academic achievement in physics.
%\item Member Sigma Pi Sigma physics honors society since 1996.
%\item Presidential Scholar, 1993 - 1997, awarded for general academic achievement.
\end{list}
\vspace{\baselineskip}

\noindent{\Large \underline{Leadership Roles:}}\\
\begin{list}{\labelitemi}
            {\setlength{\itemsep}{0.0in}\setlength{\parsep}{0.0in}
             \addtolength{\parskip}{-0.5in}}
\item Coordinator (2015-2016)  NO$\nu$A offline production group.
\item Deputy-leader (2012 - 2013), and leader (2014) NO$\nu$A module factory quality assurance team.
\item Co-convener (2010 - 2012) \dzero~low-mass Higgs group.
\item Leader (2010 - 2012) \dzero~$ZH\to\ell\ell b\bar{b}$ group.
\item Co-leader (2006 - 2008) \dzero~silicon microstrip tracker operations group.
\end{list}
\vspace{\baselineskip}

\noindent{\Large \underline{Employment History:}} \\
\begin{list}{\labelitemi}
            {\setlength{\itemsep}{0.0in}\setlength{\parsep}{0.0in}
             \addtolength{\parskip}{-0.5in}}
\item Research Associate, University of Minnesota, Minneapolis, MN (2012 - present)
\item Research Associate, Fermi National Accelerator Laboratory, Batavia, IL (2006 - 2012)
\item Graduate Research Assistant, Stony Brook University, Stony Brook, NY (2000 - 2006) %\\
%\item Teaching Assistant, Stony Brook University, Stony Brook, NY (1997-1998) %\\
%\item Undergraduate Research Assistant, Franklin and Marshall College, Lancaster, PA (1996 - 1997)
%\item Undergraduate Teaching Assistant, Franklin and Marshall College, Lancaster, PA (1995 - 1997)
%\item Undergraduate Physics Tutor, Franklin and Marshall College, Lancaster, PA (1995 - 1997)
\end{list}
\vspace{\baselineskip}

%\noindent{\Large \underline{Research Interests:}} Electroweak symmetry breaking,
%Precision electroweak measurements, QCD, Grid computing, Detector development,
%construction and operations. \\

%%\noindent{\Large \underline{Professional Affiliations:}} American Physical Society \\

%%\pagebreak
\noindent{\Large \underline{Experience:}} \\

\noindent{\bf Searches for the Source of Electroweak Symmetry Breaking} \\
\begin{list}{\labelitemi}
            {\setlength{\itemsep}{0.0in}\setlength{\parsep}{0.0in}
             \addtolength{\parskip}{-0.5in}}

\item Co-convener \dzero~low-mass Higgs group, consisting of roughly
forty members.  Under my leadership, obtained first evidence for $VZ$ production 
with $Z\to b{\bar b}$ (I was primary\ analyzer for the $\ell\ell b{\bar b}$ input to
this result), and in combination with CDF, of $H\to b{\bar b}$.  Led effort to publish
final \dzero~Higgs search papers on full Run~II data set of 9.7 fb$^{-1}$.
\item Leader of \dzero~$ZH\to\ell\ell b{\bar b}$ analysis team, consisting
of three students, five postdocs and four senior scientists.  Primary developer of
analysis software for all stages of analysis. Pioneered initial development of multivariate 
discriminant to suppress $t{\bar t}$ background, and supervised student in its completion.
\item I provided the \dzero~$\ell\ell b{\bar b}$ inputs to articles on
  combined \dzero, and combined Tevatron constraints on the spin and parity of the Higgs
  boson.
\item Dissertation based on search for technicolor (an alternative
to the standard model description of electroweak symmetry breaking) in
the $\rho_{TC} \to W \pi_{TC} \to \mu\nu b{\overline b}/ b{\overline c}$
channel.  Designed a random grid search to obtain separately optimized analyses
for each of 20 mass hypotheses.
%\indent Undergraduate honors thesis based on calculations of
%production cross sections for a Higgs boson plus jets in gluon fusion
%processes.  Work resulted in two papers, one for a standard model
%Higgs and one for a pseudoscalar Higgs boson, as predicted by
%multi-Higgs doublet models such as supersymmetry.
\end{list}
%\vspace{\baselineskip}


\noindent{\bf Research in Neutrino Physics}\\
\begin{list}{\labelitemi}
            {\setlength{\itemsep}{0.0in}\setlength{\parsep}{0.0in}
             \addtolength{\parskip}{-0.5in}}
\item Studied feasability of low energy electron reconstruction in NO$\nu$A experiment.
\item Performed first data-Monte Carlo comparisons in NO$\nu$A far
  detector.  Compared factory measurements of fiber quality to
  performance of installed detector.
\end{list}

\noindent{\bf Detector Construction, Commissioning, and Operations}\\
 
\begin{list}{\labelitemi}
            {\setlength{\itemsep}{0.0in}\setlength{\parsep}{0.0in}
             \addtolength{\parskip}{-0.5in}}
\item Deputy-leader and later leader of the  NO$\nu$A module
 factory quality assurance team, consisting of two postdocs, two
 computer scientists and several undergraduates.  Factory employed
 several hundred undergraduates to construct more than 11,000 modules
 over the course of three years.  Developed software tools for
 monitoring waste rates during module construction.  Maintained factory 
 database used to track module construction and status.  Initiated use of
 electronic logbook to track activities at module factory.  Determined
 acceptable levels of defects for modules used in the NO$\nu$A near and far
 detectors.  
\item Jointly led silicon microstrip tracker (SMT) operations group.
  Streamlined operations and eliminated leading causes of deadtime
  from SMT.  Developed tool for real-time monitoring of SMT hit
  efficiency.  Planned and led a major repair effort to
  recover disabled SMT sensors on a tight schedule.  Key player in
  followup efforts during subsquent shutdowns.  Resulted in highest
  fraction of active SMT channels in \dzero~history an established new
  trend of stability in SMT operations.
\item On call expert for \dzero~fiber tracker and silicon microstrip
  tracker.  Responsibilities included rapid diagnosis and
  resolution of detector problems, and training of shifters.
\item Developed firmware for data transmission in \dzero~central track
  trigger (CTT). Primary player in timing CTT into \dzero~trigger and
  data-acquisition system.
\item Partcipated in design, construction and calibration of
  \dzero~forward preshower (FPS) calibration system.  Developed DAQ
  and analysis software for calibration test stand.
\item Initiated and led effort to verify optical fiber connectivity
  and mapping, for installed FPS detector.
\end{list}
%\vspace{\baselineskip}

\noindent{\bf Computing}\\

\begin{list}{\labelitemi}
            {\setlength{\itemsep}{0.0in}\setlength{\parsep}{0.0in}
             \addtolength{\parskip}{-0.5in}}
\item Leader of NO$\nu$A offline production group, consisting of 
  two students and five postdocs, responsible for generation and
  processing of all data and Monte Carlo files.  Coordinated this
  effort for NO$\nu$A first and second analyses, collectively involving
  the production of more than ten million files consuming approximately
  1.5 petabytes of space. Interfaced with Fermilab Computing Division for
  monitoring health of grid resources, and routine processing of NO$\nu$A data.
\item Wrote and maintain general purpose grid utilties adopted
  as a standard tool by whole collaboration.
\item Responsible for reconstruction of NO$\nu$A Monte Carlo files.
\item Developed tools for production of NO$\nu$A calibration input files, and
  handed off to Fermilab Computing Division
\item Member of small team of core administrators for \cluedo\ Linux
  cluster.  During my term in this role, \cluedo~was adopted by
  \dzero~collaboration as its primary desktop computing solution, and
  grew to encompass hundreds of PCs and servers of widely varying
  configurations.
\end{list}
%\vspace{\baselineskip}

\noindent{\bf Teaching}\\

\begin{list}{\labelitemi}
            {\setlength{\itemsep}{0.0in}\setlength{\parsep}{0.0in}
             \addtolength{\parskip}{-0.5in}}
\item Instructor for the ``Energy and the Environment'' introductory
  physics course for approximately 100 undergraduate students at the
  University of Minnesota (2014).  Presented material to students in a
  clear and compelling way, making frequent connections to current
  events.
\item As leader of the \dzero~$\ell\ell b{\bar b}$ analysis team,
  supervised three students (Jiaming Yu, Peng Jiang, Suneel Dutt) in
  their thesis research.
\item Local organizer and moderator at University of Minnesota for
  Quarknet masterclass for high-school students.  Local moderator for
  same at Fermilab.  Students are exposed to basic concepts in
  particle physics, and asked to perform a simple analysis based on
  scanning event displays.  They then present results over video to
  other students from several other locations around the world.
%\item Laboratory instructor for introductory physics at Stony Brook University (1997-1998),
%and at Franklin and Marshall College (1996 - 1997).
%
%\item Introductory Physics tutor at Franklin and Marshall College (1994-1997).
%
%\item Intermediate Electricity and Magnetism  tutor at Franklin and Marshall College (1996).
\end{list}
%\vspace{\baselineskip}

\noindent{\bf Outreach}\\
\begin{list}{\labelitemi}
            {\setlength{\itemsep}{0.0in}\setlength{\parsep}{0.0in}
             \addtolength{\parskip}{-0.5in}}
\item Participant in Fermilab Ask-a-Scientist program.
\item Gave tours of Fermilab facilities to groups of diverse backgrounds, from visiting scientists
to secondary-school students, including as part of Fermilab Saturday Morning Physics program.
\item Judge, Department of Energy sponsored 2009 SERCh competition for undergraduate research.
\end{list}
%\vspace{\baselineskip}

\noindent{\bf Committees}\\
\begin{list}{\labelitemi}
            {\setlength{\itemsep}{0.0in}\setlength{\parsep}{0.0in}
             \addtolength{\parskip}{-0.5in}}
\item Executive Steering Committee, University of Minnesota Postdoctoral Association
\item \dzero~Advisory Council, elected by the \dzero~collaboration.  Provided
  advice to spokespersons on a variety of topics of import to the collaboration.
\item Internal Review Board for $t{\bar t}$ cross section measurements and
  searches for exotic particles using $t{\bar t}$ events.
\end{list}
%\vspace{\baselineskip}

%\newpage
\noindent{\Large \underline{Publications:}} \\

I am an author on all Run II publications from the \dzero\ collaboration. This includes
more than 340 articles.  A complete list of articles may be found at:
http://www-d0.fnal.gov/d0\_publications/d0\_pubs\_list\_bydate.html

I am also on the initial author list for the NO$\nu$A collaboration.\\

\input svdpub.tex
%\vspace{\baselineskip}

%\input svdproc.tex
%
\newpage
\noindent{\Large \underline{Selected Presentations:}}

\begin{enumerate}
\item Rice University, HEP Seminar, November 2015: 
``First Oscillation Results from NO$\nu$A''
\item Oklahoma State University, HEP Seminar and Colloquium, April 2015:
``The NO$\nu$A Experiment''
\item Imperial College London, HEP Seminar, June 2014:
``The NO$\nu$A Experiment''
\item  Meeting of the American Physical Society, Denver, CO.  May 2013:
``Status of the NO$\nu$A Experiment''
\item University of Illinois at Chicago, HEP Seminar, September 2012:
``Evidence for a New Particle Produced in Association with Weak Bosons 
  and Decaying to $b\bar{b}$ Pairs in Higgs Searches at the Tevatron''
\item 36th International Conference on High Energy Physics, July 2012:
``Searches for $H\to b\bar{b}$ at \dzero''
\item Fermilab Joint Experimental-Theoretical Physics Seminar, March 2012:
``Latest Higgs Results from \dzero''
\item Les Rencontres de Physique de la Vallée d'Aoste, March 2012:
``Low Mass Higgs Searches at the Tevatron''
\item University of California at Davis, Colloquium, October 2011:
``Hunting the Higgs at the Tevatron''
\item 19th International Conference on Supersymmetry and Unification of Fundamental Interactions,
Fermilab, September 2011:
``Tevatron Combination of SM Higgs Searches and 4th Generation Limits''.
\item 23rd Rencontres de Blois, Blois France, June 2011:
``Search for Low Mass Higgs at the Tevatron.''
\item Indiana University HEP Seminar, November 2010:
``Hunting for the Higgs at \dzero''
\item 30th Physics in Collision Symposium, Karlsruhe, Germany.  September 2010:
``Higgs Searches at the Tevatron.''
\item The University of \dzero~Lecture Series, May 2010: ``Making Tracks at \dzero''
\item 20th Hadron Collider Physics Symposium, Evian, France. November 2009:
``Low Mass Higgs at the Tevatron.''
\item Pheno 2009 Symposium, Madison, Wisconsin, May 2009: 
``Search for $ZH\to e^+e^-b{\bar b}$ in $p\bar{p}$ Collisions at \dzero''
\item 17th International Workshop on Vertex Detectors, Uto Island, Sweden.  July 2008:
``Radiation Damage Study of the \dzero\ Silicon Microstrip Tracker''
\item European Physical Society Conference on High Energy Physics, Manchester, England.  July 2007:
``Searches for Resonant Higgs Production at \dzero.''
%\item Meeting of the American Physical Society, Dallas, TX.  April 2006:
%``A Search for Technicolor Production at \dzero.''
%\item Meeting of the American Physical Society, Dallas, TX.  April 2006:
%``A Search for WH Associated Production at \dzero.''
%\item Meeting of the American Physical Society, Denver, CO.  May 2004:
% ``$W(\to \mu\nu)b{\overline b}$ Production at \dzero.''
%\item Meeting of the American Physical Society, Denver, CO.  May 2004:
% ``$W(\to e\nu)b{\overline b}$ Production at \dzero.''
%\item Meeting of the American Physical Society, Philadelphia, PA.  April 2003:
% ``The Level 1 Central Track Trigger at \dzero.''
\end{enumerate}

\end{document}
